\documentclass{amsart}

\usepackage{amssymb}

\def\vec#1{\mathchoice{\mbox{\boldmath$\displaystyle#1$}}
	{\mbox{\boldmath$\textstyle#1$}}
	{\mbox{\boldmath$\scriptstyle#1$}}
	{\mbox{\boldmath$\scriptscriptstyle#1$}}}
\newcommand{\floor}[1]{\left\lfloor #1\right\rfloor}
\newcommand{\ceil}[1]{\left\lceil #1\right\rceil}
\newcommand{\F}{\mathbb{F}}
\newcommand{\N}{\mathbb{N}}

\newcommand{\Input}{\textbf{Input:}}
\newcommand{\Output}{\textbf{Output:}}

\newcommand{\field}{\normalfont\texttt{gf256}}
\newcommand{\transforms}{\normalfont\texttt{transforms}}
\newcommand{\hermite}{\normalfont\texttt{hermite}}
\newcommand{\encode}{\normalfont\texttt{encode}}

\newcommand{\XtoN}{LCHToNewton}
\newcommand{\NtoX}{NewtonToLCH}
\newcommand{\XtoL}{LCHEval}
\newcommand{\LtoX}{LCHInterp}
\newcommand{\lsXtoL}{LowSpaceLCHEval}
\newcommand{\lsLtoX}{LowSpaceLCHInterp}
\newcommand{\XtoM}{LCHToMonomial}
\newcommand{\MtoX}{MonomialToLCH}
\newcommand{\Taylor}{TaylorExpansion}
\newcommand{\InverseTaylor}{InverseTaylorExpansion}

\makeatletter
\newcommand*{\bdiv}{%
	\nonscript\mskip-\medmuskip\mkern5mu%
	\mathbin{\operator@font div}\penalty900\mkern5mu%
	\nonscript\mskip-\medmuskip
}
\makeatother

\let\originalleft\left
\let\originalright\right
\renewcommand{\left}{\mathopen{}\mathclose\bgroup\originalleft}
\renewcommand{\right}{\aftergroup\egroup\originalright}

\newenvironment{inputs}{\Input\par\begin{tabular}{@{}rcl}}{\end{tabular}}
\newenvironment{outputs}{\Output\par}{}

\setlength{\parindent}{0pt}
\setlength{\parskip}{1ex plus 0.5ex minus 0ex}

\begin{document}
	
\title{\lowercase{ftoc2}: Fast transforms over characteristic 2}
\maketitle

\section{\field{}: Implementation of arithmetic in $\F_{2^8}$}

The module \field{} implements field arithmetic for $\F_{2^8}$. We let
$\F_{2^8}=\F_2(\alpha)$ for a root $\alpha$ of $x^8+x^4+x^3+x^2+1$, and
$\{\beta_0,\dotsc,\beta_7\}$ be the basis of the extension obtained by letting
$\beta_7=\alpha^5$ and $\beta_i=\beta^2_{i+1}-\beta_{i+1}$ for
$i\in\{0,\dotsc,6\}$. This construction yields a so-called Cantor basis (see
\cite[Appendix~A]{gao2010}). Define maps $[{}\cdot{}]_k:\N\rightarrow\{0,1\}$
for $k\in\N$ by
\begin{equation*}
	i=\sum_{k\in\N}2^k[i]_k
	\quad\text{for $i\in\N$}.
\end{equation*}
The field $\F_{2^8}$ is enumerated as $\{\omega_0,\dotsc,\omega_{255}\}$, with
\begin{equation*}
	\omega_i=\sum^7_{k=0}[i]_k\beta_k
	\quad\text{for $i\in\{0,\dotsc,255\}$}.
\end{equation*}
Then $\F_{2^d}=\{\omega_0,\dotsc,\omega_{2^d-1}\}$ for $d\in\{1,2,4,8\}$.

An element $\omega_i\in\F_{2^8}$ is represented by the byte
$[i]_7[i]_6\dotsb[i]_0$. Thus, addition of field elements corresponds to
computing the XOR of bytes. Logarithm and exponent tables are used for
multiplication. The array $\texttt{LOG}$ contains logarithms base $\alpha$:
$\texttt{LOG}[i]$ is the byte that corresponds to $\log_\alpha\omega_i$ for
$i\in\{1,\dotsc,255\}$, and $\texttt{LOG}[0]$ is arbitrarily set to zero.
Similarly, $\texttt{EXP}[i]$ corresponds to $\alpha^i$ for
$i\in\{0,\dotsc,254\}$. 

\subsection{Functions}\

\texttt{unsigned char mul(unsigned char x, unsigned char y)}

\begin{inputs}
	\texttt{x} & - & field element, \\
	\texttt{y} & - & field element.
\end{inputs}

\begin{outputs}
	The product of the two elements $x,y\in\F_{2^8}$ that correspond to the inputs.
\end{outputs}

\section{\transforms{}: Transforms over $\F_{2^8}$}

The module \transforms{} implements several of the fast transforms proposed
in~\cite{gao2010,lin2014,lin2016a,coxon2018b}. Define polynomials
\begin{equation*}
	N_i
	=
	\prod^{i-1}_{j=0}
	\left(x-\omega_j\right)
	\quad\text{and}\quad
	X_i
	=
	\prod^7_{k=0}
	\prod^{2^k[i]_k-1}_{j=0}
	\left(x-\omega_j\right)
\end{equation*}
for $i\in\{0,\dotsc,255\}$. The polynomials $N_i$ are the Newton polynomials
associated with the enumeration $\{\omega_0,\dotsc,\omega_{255}\}$ of
$\F_{2^8}$. The polynomials $X_i$ were introduced by Lin, Chung and Han
\cite{lin2014}, and referred to hereafter as the LCH basis polynomials.

\subsection{Functions}\

\texttt{unsigned int log2(unsigned int n)}

Ceiling of base two logarithm.

\begin{inputs}
	\texttt{n} & - & an integer.
\end{inputs}

\begin{outputs}
	The ceiling of $\log_2 n$.
\end{outputs}

\texttt{void \NtoX(int l, unsigned char[:]\ a)}

Newton basis to LCH basis conversion.

\begin{inputs}
	\texttt{l} & - & length of the polynomial, \\
	\texttt{a} & - & coefficient array.
\end{inputs}

\begin{outputs}
	If initially $\texttt{a}[i]=f_i\in\F_{2^8}$ for $i\in\{0,\dotsc,\ell-1\}$, then
	on termination $\texttt{a}[i]=h_i\in\F_{2^8}$ for $i\in\{0,\dotsc,\ell-1\}$ such
	that
	\begin{equation}\label{eqn:newton-lch}
		\sum^{\ell-1}_{i=0}f_iN_i
		=\sum^{\ell-1}_{i=0}h_iX_i.
	\end{equation}
\end{outputs}

\texttt{void \XtoN(int l, unsigned char[:]\ a)}

LCH basis to Newton basis conversion.

\begin{inputs}
	\texttt{l} & - & length of the polynomial, \\
	\texttt{a} & - & coefficient array.
\end{inputs}

\begin{outputs}
	If initially $\texttt{a}[i]=h_i\in\F_{2^8}$ for $i\in\{0,\dotsc,\ell-1\}$, then
	on termination $\texttt{a}[i]=f_i\in\F_{2^8}$ for $i\in\{0,\dotsc,\ell-1\}$ such
	that \eqref{eqn:newton-lch} holds.
\end{outputs}

\texttt{void \LtoX(unsigned char shift, int c, int l, unsigned char[:]\ a)}

Interpolation on the LCH basis.

\begin{inputs}
	\texttt{shift} & - & shift parameter,                \\
	\texttt{c} & - & number of evaluations provided, \\
	\texttt{l} & - & length of the polynomial,       \\
	\texttt{a} & - & coefficient array of length $2^{\ceil{\log_2\ell}}$.
\end{inputs}

\begin{outputs}
	Given $\texttt{shift}=\lambda\in\F_{2^8}$, $\texttt{a}[i]=f_i\in\F_{2^8}$ for
	$i\in\{0,\dotsc,c-1\}$, and $\texttt{a}[i]=h_i\in\F_{2^8}$ for
	$i\in\{c,\dotsc,\ell-1\}$, the function terminates with
	$\texttt{a}[i]=h_i\in\F_{2^8}$ for $i\in\{0,\dotsc,c-1\}$ such that
	\begin{equation}\label{eqn:lagrange-lch}
		\sum^{\ell-1}_{i=0}
		h_i
		X_i\left(\omega_j+\lambda\right)
		=
		f_j
		\quad
		\text{for $j\in\{0,\dotsc,c-1\}$}.
	\end{equation}
\end{outputs} 

\texttt{void \XtoL(unsigned char shift, int c, int l, unsigned char[:]\ a)}

Evaluation on the LCH basis.

\begin{inputs}
	\texttt{shift} & - & shift parameter,                  \\
	\texttt{c} & - & number of evaluations to compute, \\
	\texttt{l} & - & length of the polynomial,         \\
	\texttt{a} & - & coefficient array of length
	$\max(\ell,2^{\ceil{\log_2c}})$.
\end{inputs}

\begin{outputs}
	Given $\texttt{shift}=\lambda\in\F_{2^8}$ and $\texttt{a}[i]=h_i\in\F_{2^8}$ for
	$i\in\{0,\dotsc,\ell-1\}$, the function terminates with
	$\texttt{a}[i]=f_i\in\F_{2^8}$ for $i\in\{0,\dotsc,c-1\}$ such that
	\eqref{eqn:lagrange-lch} holds.
\end{outputs}

\texttt{void \lsXtoL(unsigned char shift, int c, int l, unsigned char[:]~a)}

In-place evaluation on the LCH basis.

\begin{inputs}
	\texttt{shift} & - & shift parameter,                  \\
	\texttt{c}     & - & number of evaluations to compute, \\
	\texttt{l}     & - & length of the polynomial,         \\
	\texttt{a}     & - & coefficient array.
\end{inputs}

\begin{outputs}
	Given $\texttt{shift}=\lambda\in\F_{2^8}$, $\texttt{a}[i]=h_i\in\F_{2^8}$ for
	$i\in\{0,\dotsc,\ell-1\}$, and $\texttt{a}[i]\in\F_{2^8}$ for
	$i\in\{c,\dotsc,\ell-1\}$ (i.e, their value does not matter), the function terminates with
	$\texttt{a}[i]=f_i\in\F_{2^8}$ for $i\in\{0,\dotsc,c-1\}$ such that
	\eqref{eqn:lagrange-lch} holds, while $\texttt{a}[c],\dotsc,\texttt{a}[\ell-1]$
	retain their initial values if $c<\ell$.
\end{outputs}

\texttt{void \lsLtoX(unsigned char shift, int c, int l, unsigned char[:]~a)}

In-place interpolation on the LCH basis.

\begin{inputs}
	\texttt{shift} & - & shift parameter,                \\
	\texttt{c}     & - & number of evaluations provided, \\
	\texttt{l}     & - & length of the polynomial,       \\
	\texttt{a}     & - & coefficient array.
\end{inputs}

\begin{outputs}
	Given $\texttt{shift}=\lambda\in\F_{2^8}$, $\texttt{a}[i]=f_i\in\F_{2^8}$ for
	$i\in\{0,\dotsc,c-1\}$, and $\texttt{a}[i]=h_i\in\F_{2^8}$ for
	$i\in\{c,\dotsc,\ell-1\}$, the function terminates with
	$\texttt{a}[i]=h_i\in\F_{2^8}$ for $i\in\{0,\dotsc,c-1\}$ such that
	\eqref{eqn:lagrange-lch} holds, while $\texttt{a}[c],\dotsc,\texttt{a}[\ell-1]$ 
	retain their initial values.
\end{outputs} 

\texttt{void \Taylor(int t, int l, unsigned char[:]\ a)}

Generalised Taylor expansion at $x^{2^t}-x$.

\begin{inputs}
	\texttt{t} & - & exponent,                 \\
	\texttt{l} & - & length of the polynomial, \\
	\texttt{a} & - & coefficient array.
\end{inputs}

\begin{outputs}
	If initially $\texttt{a}[i]=f_i\in\F_{2^8}$ for $i\in\{0,\dotsc,\ell-1\}$, then
	on termination $\texttt{a}[i]=h_i\in\F_{2^8}$ for $i\in\{0,\dotsc,\ell-1\}$ such
	that
	\begin{equation}\label{eqn:taylor}
		\sum^{\ell-1}_{i=0}f_ix^i
		=
		\sum^{\ell-1}_{i=0}
		h_i
		x^{i-2^t\floor{i/2^t}}
		\left(x^{2^t}-x\right)^{\floor{i/2^t}}.
	\end{equation}
\end{outputs}

\texttt{void InverseTaylorExpansion(int t, int l, unsigned char[:]\ a)}

Inverse generalised Taylor expansion at $x^{2^t}-x$.

\begin{inputs}
	\texttt{t} & - & exponent,                 \\
	\texttt{l} & - & length of the polynomial, \\
	\texttt{a} & - & coefficient array.
\end{inputs}

\begin{outputs}
	If initially $\texttt{a}[i]=h_i\in\F_{2^8}$ for $i\in\{0,\dotsc,\ell-1\}$, then
	on termination $\texttt{a}[i]=f_i\in\F_{2^8}$ for $i\in\{0,\dotsc,\ell-1\}$ such
	that \eqref{eqn:taylor} holds.
\end{outputs}

\texttt{void \MtoX(int l, unsigned char[:]\ a)}

Monomial basis to LCH basis conversion.

\begin{inputs}
	\texttt{l} & - & length of the polynomial, \\
	\texttt{a} & - & coefficient array.
\end{inputs}

\begin{outputs}
	If initially $\texttt{a}[i]=f_i\in\F_{2^8}$ for $i\in\{0,\dotsc,\ell-1\}$, then
	on termination $\texttt{a}[i]=h_i\in\F_{2^8}$ for $i\in\{0,\dotsc,\ell-1\}$ such
	that $\sum^{\ell-1}_{i=0}f_ix^i=\sum^{\ell-1}_{i=0}h_iX_i$.
\end{outputs}

\texttt{void \XtoM(int l, unsigned char[:]\ a)}

LCH basis to monomial basis conversion.

\begin{inputs}
	\texttt{l} & - & length of the polynomial, \\
	\texttt{a} & - & coefficient array.
\end{inputs}

\begin{outputs}
	If initially $\texttt{a}[i]=h_i\in\F_{2^8}$ for $i\in\{0,\dotsc,\ell-1\}$, then
	on termination $\texttt{a}[i]=f_i\in\F_{2^8}$ for $i\in\{0,\dotsc,\ell-1\}$ such
	that $\sum^{\ell-1}_{i=0}h_iX_i=\sum^{\ell-1}_{i=0}f_ix^i$.
\end{outputs}

\texttt{void TaylorExpansionHighCoeffs(int t, int c, int l, unsigned char[:]\
	a)}

High coefficients of the generalised Taylor expansion at $x^{2^t}-x$.

\begin{inputs}
	\texttt{t} & - & exponent,                 \\
	\texttt{c} & - & cut-off,                  \\
	\texttt{l} & - & length of the polynomial, \\
	\texttt{a} & - & coefficient array.
\end{inputs}

\begin{outputs}
	If initially $\texttt{a}[i]=f_i\in\F_{2^8}$ for ${i\in\{c,\dotsc,\ell-1\}}$,
	then on termination $\texttt{a}[i]=h_i\in\F_{2^8}$ for $i\in\{c,\dotsc,\ell-1\}$
	such that
	\begin{equation}\label{eqn:taylor}
		\sum^{\ell-1}_{i=c}f_ix^i
		=
		\sum^{\ell-1}_{i=0}
		h_i
		x^{i-2^t\floor{i/2^t}}
		\left(x^{2^t}-x\right)^{\floor{i/2^t}},
	\end{equation}
	for some $h_0,\dots,h_{c-1}\in\F_{2^8}$, while
	$\texttt{a}[0],\dotsc,\texttt{a}[c-1]$ retain their initial values.
\end{outputs}

\texttt{void \MtoX HighCoeffs(int c, int l, unsigned char[:]\ a)}

Monomial basis to LCH basis conversion.

\begin{inputs}
	\texttt{c} & - & cut-off,                  \\
	\texttt{l} & - & length of the polynomial, \\
	\texttt{a} & - & coefficient array.
\end{inputs}

\begin{outputs}
	If initially $\texttt{a}[i]=f_i\in\F_{2^8}$ for ${i\in\{c,\dotsc,\ell-1\}}$,
	then on termination $\texttt{a}[i]=h_i\in\F_{2^8}$ for $i\in\{c,\dotsc,\ell-1\}$
	such that
	$\sum^{\ell-1}_{i=c}f_ix^i=\sum^{\ell-1}_{i=0}h_iX_i$ for some
	$h_0,\dots,h_{c-1}\in\F_{2^8}$, while $\texttt{a}[0],\dotsc,\texttt{a}[c-1]$
	retain their initial values.
\end{outputs}

\section{\hermite{}: Hermite interpolation and evaluation over $\F_{2^8}$}

The module \hermite{} is an implementation of the Hermite interpolation and
evaluation algorithms proposed in \cite{coxon2018}, specialised to $\F_{2^8}$
and its subfields. These algorithms reduce instances of the Hermite problems to
instances of regular (i.e., without the presence of derivatives) interpolation
and evaluation problems. We implement these reduction, and use the \transforms{}
module to solve the problems they admit. The algorithms used in the module are
more general than their counterparts in~\cite{coxon2018}, since they include an
additional parameter $\ell$ that accounts for polynomial length.

For $i\in\N$, let $D^i:\F_{2^8}[x]\rightarrow\F_{2^8}[x]$ be the map that sends
$f\in\F_{2^8}[x]$ to the coefficient of $y^i$ in $f(x+y)\in\F_{2^8}[x][y]$,
called the $i$th Hasse derivative on $\F_{2^8}[x]$.

\subsection{Functions}\

\texttt{void Evaluate(int c, int l, unsigned char[:]\ a)}

Algorithm 1.

\begin{inputs}
	\texttt{c} & - & number of evaluations to compute, \\
	\texttt{l} & - & length of the polynomial,         \\
	\texttt{a} & - & coefficient array.
\end{inputs}

\begin{outputs}
	If initially $\texttt{a}[i]=f_i\in\F_{2^8}$ for $i\in\{0,\dotsc,\ell-1\}$, then
	on termination $\texttt{a}[i]=f(\omega_i)\in\F_{2^8}$ for
	$f=\sum^{\ell-1}_{i=0}f_ix^i$ and $i\in\{0,\dotsc,c-1\}$.
\end{outputs}

\texttt{void Interpolate(int c, int l, unsigned char[:]\ a)}

Algorithm 3.

\begin{inputs}
	\texttt{c} & - & number of evaluations provided, \\
	\texttt{l} & - & length of the polynomial,       \\
	\texttt{a} & - & coefficient array.
\end{inputs}

\begin{outputs}
	If initially $\texttt{a}[i]=h_i\in\F_{2^8}$ for $i\in\{0,\dotsc,c-1\}$, and
	$\texttt{a}[i]=f_i\in\F_{2^8}$ for $i\in\{c,\dotsc,\ell-1\}$, then on
	termination $\texttt{a}[i]=f_i\in\F_{2^8}$ for $i\in\{0,\dotsc,c-1\}$ such that
	$f=\sum^{\ell-1}_{i=0}f_ix^i$ satisfies $f(\omega_i)=h_i$ for
	$i\in\{0,\dotsc,c-1\}$.
\end{outputs}

\texttt{void PrepareLeft(int d, int n, int c, int l, unsigned char[:]\ a)}

The function \textsf{PrepareLeft} from Algorithm 2.

\texttt{void PrepareRight(int d, int n, int c, int l, unsigned char[:]\ a)}

The function \textsf{PrepareRight} from Algorithm 4.

\texttt{void HermiteEvaluate(int d, int n, int c, int l, unsigned char[:]\ a)}

Algorithm 2.

\begin{inputs}
	\texttt{d} & - & field degree,                     \\
	\texttt{c} & - & number of evaluations to compute, \\
	\texttt{l} & - & length of the polynomial,         \\
	\texttt{a} & - & coefficient array of length
	$2^{\lceil\log_2\lceil\ell/2^d\rceil\rceil+d}$.
\end{inputs}

\begin{outputs}
	If initially $\texttt{a}[i]=f_i\in\F_{2^8}$ for $i\in\{0,\dotsc,\ell-1\}$, then
	on termination
	$\texttt{a}[i]=(D^{i\bdiv{2^d}}f)(\omega_{i\bmod{2^d}})\in\F_{2^8}$ for
	$f=\sum^{\ell-1}_{i=0}f_ix^i$ and $i\in\{0,\dotsc,c-1\}$.
\end{outputs}

\texttt{void HermiteInterpolate(int d, int n, int c, int l, unsigned char[:]\
	a)}

Algorithm 4.

\begin{inputs}
	\texttt{d} & - & field degree,                   \\
	\texttt{c} & - & number of evaluations provided, \\
	\texttt{l} & - & length of the polynomial,       \\
	\texttt{a} & - & coefficient array of length
	$2^{\lceil\log_2\lceil\ell/2^d\rceil\rceil+d}$.
\end{inputs}

\begin{outputs}
	If initially $\texttt{a}[i]=h_i\in\F_{2^8}$ for $i\in\{0,\dotsc,c-1\}$, and
	$\texttt{a}[i]=f_i\in\F_{2^8}$ for $i\in\{c,\dotsc,\ell-1\}$, then on
	termination $\texttt{a}[i]=f_i\in\F_{2^8}$ for $i\in\{0,\dotsc,c-1\}$ such that
	$f=\sum^{\ell-1}_{i=0}f_ix^i$ satisfies
	$(D^{i\bdiv{2^d}}f)(\omega_{i\bmod{2^d}})=h_i$ for $i\in\{0,\dotsc,c-1\}$.
\end{outputs}

\section{\encode{}: Systematic encoding of multiplicity codes over $\F_{2^8}$}

The module \encode{} is an implementation of the systematic encoding algorithm
for multiplicity codes proposed in \cite{coxon2017}, specialised to codes over
$\F_{2^8}$ and its subfields. To encode, the algorithm solves a multivariate
Hermite interpolation problem, followed by a multivariate Hermite evaluation
problem. Both these problems are solved by algorithms that reduce the
multivariate problems to instances of the univariate problems. The module
\hermite{} is used to solve the univariate problems. The reduction also relies
on the use of a multivariate Newton-like basis. The module \transforms{} is used to
perform conversions between this basis and the monomial basis.

The ring of polynomials over $\F_q$ in indeterminates $x_1,\dotsc,x_m$ is
denoted by $\F_q[\vec{x}]=\F_q[x_1,\dotsc,x_m]$. Define
$\vec{x}^{\vec{i}}=x^{i_1}_1\dotsm x^{i_m}_m$ for
$\vec{i}=(i_1,\dotsc,i_m)\in\N^m$. For $\vec{i}\in\N^m$, let
$D^{\vec{i}}:\F_q[\vec{x}]\rightarrow\F_q[\vec{x}]$ be the map that sends
$f\in\F_q[\vec{x}]$ to the coefficient of $\vec{y}^{\vec{i}}$ in
$f(\vec{x}+\vec{y})\in\F_q[\vec{x}][\vec{y}]$, called the $\vec{i}$th Hasse
derivative on $\F_q[\vec{x}]$. Let $\F_q[\vec{x}]_\ell$ denote the space of
polynomials in $\F_q[\vec{x}]$ of total degree less than $\ell$. The weight of a
vector $\vec{i}\in\N^m$, denoted $\left|\vec{i}\right|$, is defined to be the
sum of its entries. Define
$S_{m,s}=\{\vec{s}\in\N^m\mid\left|\vec{s}\right|<s\}$ for positive $s\in\N$.
Then for $q$ a prime power and positive $m,\ell,s\in\N$ such that $\ell\leq sq$,
the corresponding multiplicity code is
\begin{equation*}
	\mathrm{Mult}(q,m,\ell,s)
	=
	\left\{
		\left(
			\left(
				(D^{\vec{s}}f)(\vec{\omega})
			\right)_{\vec{s}\in S_{m,s}}
		\right)_{\vec{\omega}\in\F^m_q}
		\mid
		f\in\F_q[\vec{x}]_\ell
	\right\}.
\end{equation*}

This module is limited to multiplicity codes with $q=2^d$ for some
$d\in\{1,2,4,8\}$. Consequently, $q$ is assumed to be of this form hereafter. We
index the elements of $\F^m_q$ by vectors in $\{0,\dotsc,q-1\}^m$ by defining
$\vec{\omega}_{\vec{i}}=(\omega_{i_1},\dotsc,\omega_{i_m})$ for
$\vec{i}=(i_1,\dotsc,i_m)\in\{0,\dotsc,q-1\}^m$. The class \texttt{Codeword},
which subclasses NumPy's ndarray, is used to store codewords (i.e., elements) of
a multiplicity code. In particular, a codeword is represented by a
one-dimensional ndarray of length $(qs)^m$. This is roughly $m!$ times the
length of the code, so much memory is wasted. If an array represents the
codeword corresponding to $f\in\F_q[\vec{x}]$, then the index of
$(D^{\vec{s}}f)(\vec{\omega}_{\vec{i}})$ in the array is
$\sum^m_{k=1}\left(i_k+s_kq\right)(sq)^{k-1}$, where $\vec{s}=(s_1,\dotsc,s_m)$
and $\vec{i}=(i_1,\dotsc,i_m)$. If in addition $\sum^m_{k=1}i_k+s_kq<\ell$, the
entry is part of the message of the codeword.

For $\vec{s}=(s_1,\dotsc,s_m)\in\N^m$ and
$\vec{i}=(i_1,\dotsc,i_m)\in\{0,\dotsc,q-1\}^m$, define
\begin{equation*}
	N_{\vec{i}+\vec{s}q}
	=N_{i_1}(x_1)\dotsm N_{i_m}(x_m)
	\left(x^q_1-x_1\right)^{s_1}\dotsm\left(x^q_m-x_m\right)^{s_m}.
\end{equation*}
The multivariate Newton basis of $\F_q[\vec{x}]$ defined in~\cite{coxon2017} is then equal to the set of all such polynomials.


\subsection{Functions}\

\texttt{void MonomialInterpolate(int d, int l, unsigned char[:]\ a)}

Univariate interpolation with respect to the monomial basis.

\begin{inputs}
	\texttt{d} & - & field degree,              \\
	\texttt{l} & - & length of the polynomial,  \\
	\texttt{a} & - & coefficient array.
\end{inputs}

\begin{outputs}
	If initially $\texttt{a}[i]=h_i\in\F_{2^8}$ for $i\in\{0,\dotsc,\ell-1\}$, then
	on termination $\texttt{a}[i]=f_i\in\F_{2^8}$ for $i\in\{0,\dotsc,\ell-1\}$ such
	that $f=\sum^{\ell-1}_{i=0}f_ix^i$ satisfies
	$(D^{i\bdiv{2^d}}f)(\omega_{i\bmod{2^d}})=h_i$ for $i\in\{0,\dotsc,\ell-1\}$.
\end{outputs}

\texttt{void MonomialEvaluate(int d, int l, int s, unsigned char[:]\ a)}

Univariate evaluation with respect to the monomial basis.

\begin{inputs}
	\texttt{d} & - & field degree,              \\
	\texttt{l} & - & length of the polynomial,  \\
	\texttt{s} & - & bound on derivative order, \\
	\texttt{a} & - & coefficient array.
\end{inputs}

\begin{outputs}
	If initially $\texttt{a}[i]=f_i\in\F_{2^8}$ for $i\in\{0,\dotsc,\ell-1\}$, then
	on termination
	$\texttt{a}[i]=(D^{i\bdiv{2^d}}f)(\omega_{i\bmod{2^d}})\in\F_{2^8}$ for
	$f=\sum^{\ell-1}_{i=0}f_ix^i$ and $i\in\{0,\dotsc,2^ds-1\}$.
\end{outputs}

\texttt{void SmallMonomialToNewton(int l, unsigned char[:]\ a)}

Conversion from monomial to Newton basis for $\ell\leq 2^8$.

\begin{inputs}
	\texttt{l} & - & length of the polynomial, \\
	\texttt{a} & - & coefficient array.
\end{inputs}

\begin{outputs}
	If initially $\texttt{a}[i]=f_i\in\F_{2^8}$ for $i\in\{0,\dotsc,\ell-1\}$, then
	on termination $\texttt{a}[i]=h_i\in\F_{2^8}$ such that
	$\sum^{\ell-1}_{i=0}h_iN_i=\sum^{\ell-1}_{i=0}f_ix^i$.
\end{outputs}

\texttt{void MonomialToNewton(int d, int l, unsigned char[:]\ a)}

Conversion from monomial to Newton basis.

\begin{inputs}
	\texttt{d} & - & field degree,             \\
	\texttt{l} & - & length of the polynomial, \\
	\texttt{a} & - & coefficient array.
\end{inputs}

\begin{outputs}
	If initially $\texttt{a}[i]=f_i\in\F_{2^8}$ for $i\in\{0,\dotsc,\ell-1\}$, then
	on termination $\texttt{a}[i]=h_i\in\F_{2^8}$ such that
	\begin{equation}\label{eqn:gen-newton-monomial}
		\sum^{\ell-1}_{i=0}f_ix^i
		=
		\sum^{\ell-1}_{i=0}
		h_i
		N_{i-2^d\floor{i/2^d}}
		\left(x^{2^d}-x\right)^{\floor{i/2^d}}.
	\end{equation}
\end{outputs}

\texttt{void SmallNewtonToMononial(int l, unsigned char[:]\ a)}

Conversion from Newton to monomial basis for $\ell\leq 2^8$.

\begin{inputs}
	\texttt{l} & - & length of the polynomial, \\
	\texttt{a} & - & coefficient array.
\end{inputs}

\begin{outputs}
	If initially $\texttt{a}[i]=h_i\in\F_{2^8}$ for $i\in\{0,\dotsc,\ell-1\}$, then
	on termination $\texttt{a}[i]=f_i\in\F_{2^8}$ such that
	$\sum^{\ell-1}_{i=0}f_ix^i=\sum^{\ell-1}_{i=0}h_iN_i$.
\end{outputs}

\texttt{void NewtonToMonomial(int d, int l, unsigned char[:]\ a)}

Conversion from Newton to monomial basis.

\begin{inputs}
	\texttt{d} & - & field degree,             \\
	\texttt{l} & - & length of the polynomial, \\
	\texttt{a} & - & coefficient array.
\end{inputs}

\begin{outputs}
	If initially $\texttt{a}[i]=h_i\in\F_{2^8}$ for $i\in\{0,\dotsc,\ell-1\}$, then
	on termination $\texttt{a}[i]=f_i\in\F_{2^8}$ such that
	\eqref{eqn:gen-newton-monomial} holds.
\end{outputs}

\texttt{void NewtonInterpolate(int d, int l, unsigned char[:]\ a)}

Univariate interpolation with respect to the Newton basis.

\begin{inputs}
	\texttt{d} & - & field degree,             \\
	\texttt{l} & - & length of the polynomial, \\
	\texttt{a} & - & coefficient array.
\end{inputs}

\begin{outputs}
	If initially $\texttt{a}[i]=h_i\in\F_{2^8}$ for $i\in\{0,\dotsc,\ell-1\}$, then
	on termination $\texttt{a}[i]=f_i\in\F_{2^8}$ such that
	\begin{equation}\label{eqn:univariate-newton}
		f
		=
		\sum^{\ell-1}_{i=0}
		f_i
		N_{i-2^d\floor{i/2^d}}
		\left(x^{2^d}-x\right)^{\floor{i/2^d}}
	\end{equation}
	satisfies $(D^{i\bdiv{2^d}}f)(\omega_{i\bmod{2^d}})=h_i$ for
	$i\in\{0,\dotsc,\ell-1\}$	
\end{outputs}

\texttt{void NewtonEvaluate(int d, int l, int s, unsigned char[:]\ a)}

Univariate evaluation with respect to the Newton basis.

\begin{inputs}
	\texttt{d} & - & field degree,              \\
	\texttt{l} & - & length of the polynomial,  \\
	\texttt{s} & - & bound on derivative order, \\
	\texttt{a} & - & coefficient array.
\end{inputs}

\begin{outputs}
	If initially $\texttt{a}[i]=f_i\in\F_{2^8}$ for $i\in\{0,\dotsc,\ell-1\}$, then
	on termination $\texttt{a}[i]=(D^{i\bdiv{2^d}}f)(\omega_{i\bmod{2^d}})$ for the
	polynomial $f$ defined by \eqref{eqn:univariate-newton} and
	$i\in\{0,\dotsc,2^ds-1\}$.
\end{outputs}

\texttt{InfoSet(m, l, b)}

Generator function that yields
\begin{equation*}
	\left(
	i_1+i_2b+\dotsb+i_mb^{m-1},
	i_1+\dotsb+i_m
	\right)
\end{equation*}
for all $(i_1,\dotsc,i_m)\in\N^m$ such that $i_1+\dotsb+i_m<\ell$.

\texttt{RecoverPolynomial(d, m, l, C)}

Computes the polynomial that corresponds to a message.

\begin{inputs}
	\texttt{d} & - & field degree,                                                 
	\\
	\texttt{m} & - & number of variables,                                          
	\\
	\texttt{l} & - & polynomial degree bound,                                      
	\\
	\texttt{C} & - & a codeword array containing a message, with all other entries
	set to zero.
\end{inputs}

\begin{outputs}
	Let $b=\texttt{C.b}$, $I=\{\vec{i}\in\N^m\mid\left|\vec{i}\right|<\ell\}$, and
	define $\pi:\N^m\rightarrow\N$ by
	$(i_1,\dotsc,i_m)\mapsto\sum^m_{k=1}i_kb^{k-1}$. Suppose that initially
	$\texttt{C}[i]=h_i\in\F_{2^8}$ for $i\in\pi(I)$, while all remaining entries of
	the array are zero. Then on termination, $\texttt{C}[i]=f_i\in\F_{2^8}$ for
	$i\in\pi(I)$ such that the polynomial
	\begin{equation}\label{eqn:multivariate-newton}
		f=\sum_{\vec{i}\in I}f_{\pi(\vec{i})}N_{\vec{i}}
	\end{equation}
	satisfies
	$(D^{\vec{i}\bdiv{2^d}}f)(\vec{\omega}_{\vec{i}\bmod{2^d}})=h_{\pi(\vec{i})}$
	for $\vec{i}\in I$, while all remaining entries of the array are zero.
\end{outputs}

\texttt{EvalSet(d, m, l, s, b)}

Generator function that yields
\begin{equation*}
	\left(
	i_1+i_2b+\dotsb+i_mb^{m-1},
	i_1+\dotsb+i_m,
	\floor{i_1/2^d}+\dotsb+\floor{i_m/2^d}
	\right)
\end{equation*}
for all $(i_1,\dotsc,i_m)\in\N^m$ such that $i_1+\dotsb+i_m<\ell$ and
$\floor{i_1/2^d}+\dotsb+\floor{i_m/2^d}<s$.

\texttt{EncodePolynomial(d, m, l, s, C)}

Computes the codeword that corresponds to a polynomial.

\begin{inputs}
	\texttt{d} & - & field degree,                                                 
	\\
	\texttt{m} & - & number of variables,                                          
	\\
	\texttt{l} & - & polynomial degree bound,                                      
	\\
	\texttt{s} & - & derivative order bound,                                       
	\\
	\texttt{C} & - & a codeword array containing the coefficients of a polynomial.
\end{inputs}

\begin{outputs}
	Let $b=\texttt{C.b}$, $I=\{\vec{i}\in\N^m\mid\left|\vec{i}\right|<\ell\}$,
	$C=\{\vec{i}+2^d\vec{s}\mid\vec{i}\in\{0,\dots,2^d-1\}^m,\vec{s}\in S_{m,s}\}$,
	and define $\pi:\N^m\rightarrow\N$ by
	$(i_1,\dotsc,i_m)\mapsto\sum^m_{k=1}i_kb^{k-1}$. Suppose that initially
	$\texttt{C}[i]=f_i\in\F_{2^8}$ for $i\in\pi(I)$, while all remaining entries of
	the array are zero. Then on termination, $\texttt{C}[i]=h_i\in\F_{2^8}$ for
	$i\in\pi(C)$ such that the polynomial \eqref{eqn:multivariate-newton} satisfies
	$(D^{\vec{i}\bdiv{2^d}}f)(\vec{\omega}_{\vec{i}\bmod{2^d}})=h_{\pi(\vec{i})}$
	for $\vec{i}\in C$.
\end{outputs}

\texttt{SystematicallyEncode(C)}

Systematic encoding algorithm.

\begin{inputs}
	\texttt{C} & - & a codeword array containing a message, with all other entries
	set to zero.
\end{inputs}

\begin{outputs}
	The array \texttt{C} equal to systematic encoding of the message.
\end{outputs}

\bibliographystyle{amsplain}

\begin{thebibliography}{1}
	
	\bibitem{coxon2017}
	Nicholas Coxon, \emph{Fast systematic encoding of multiplicity codes}, J.
	Symbolic
	Comput. \textbf{94} (2019), 234--254.
	
	\bibitem{coxon2018}
	Nicholas Coxon, \emph{Fast {H}ermite interpolation and evaluation over finite
		fields of characteristic two}, J. Symbolic
	Comput. \textbf{98} (2020), 270--283.
	
	\bibitem{coxon2018b}
	Nicholas Coxon, \emph{Fast transforms over finite fields of characteristic two},
	J. Symbolic	Comput., to appear.
	
	\bibitem{gao2010}
	Shuhong Gao and Todd Mateer, \emph{Additive fast {F}ourier transforms over
		finite fields}, IEEE Trans. Inform. Theory \textbf{56} (2010), no.~12,
	6265--6272.
	
	\bibitem{lin2016a}
	Sian-Jheng Lin, Tareq~Y. Al-Naffouri, Yunghsiang~S. Han, and Wei-Ho Chung,
	\emph{Novel polynomial basis with fast {F}ourier transform and its
		application to {R}eed--{S}olomon erasure codes}, IEEE Trans. Inform. Theory
	\textbf{62} (2016), no.~11, 6284--6299.
	
	\bibitem{lin2014}
	Sian-Jheng Lin, Wei-Ho Chung, and Yunghsiang~S. Han, \emph{Novel polynomial
		basis and its application to {R}eed-{S}olomon erasure codes}, 55th {A}nnual
	{IEEE} {S}ymposium on {F}oundations of {C}omputer {S}cience---{FOCS} 2014,
	IEEE Computer Soc., Los Alamitos, CA, 2014, pp.~316--325.
	
\end{thebibliography}
	
\end{document}
